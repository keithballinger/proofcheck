\documentclass{article}
\usepackage{amsmath, amsthm}

\title{Basic Number Theory}
\author{Example}
\date{}

\begin{document}
\maketitle

\section{Natural Numbers}

\begin{theorem}[Addition Commutativity]
For all natural numbers $n, m \in \mathbb{N}$, we have $n + m = m + n$.
\end{theorem}

\begin{proof}
By induction on $n$.
\end{proof}

\begin{theorem}[Zero Identity]
For all $n \in \mathbb{N}$, we have $n + 0 = n$ and $0 + n = n$.
\end{theorem}

\begin{lemma}[Successor Addition]
For all $n, m \in \mathbb{N}$, we have $\text{succ}(n) + m = \text{succ}(n + m)$.
\end{lemma}

\section{Divisibility}

\begin{definition}
We say that $a$ divides $b$ (written $a \mid b$) if there exists $k \in \mathbb{N}$ such that $b = a \cdot k$.
\end{definition}

\begin{theorem}[Divisibility Transitivity]
If $a \mid b$ and $b \mid c$, then $a \mid c$.
\end{theorem}

\begin{proof}
Suppose $a \mid b$ and $b \mid c$. Then there exist $k_1, k_2 \in \mathbb{N}$ such that $b = a \cdot k_1$ and $c = b \cdot k_2$. 
Therefore, $c = (a \cdot k_1) \cdot k_2 = a \cdot (k_1 \cdot k_2)$, so $a \mid c$.
\end{proof}

\end{document}