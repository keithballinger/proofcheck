\documentclass{article}
\usepackage{amsmath, amsthm}

\title{Prime Numbers}
\author{Example}

\begin{document}
\maketitle

\section{Prime Number Properties}

\begin{definition}[Prime Number]
A natural number $p > 1$ is prime if and only if for all $a, b \in \mathbb{N}$, if $p = a \cdot b$, then $a = 1$ or $b = 1$.
\end{definition}

\begin{theorem}[Euclid's Theorem]
There are infinitely many prime numbers.
\end{theorem}

\begin{proof}
Suppose there are finitely many primes $p_1, p_2, \ldots, p_n$. 
Consider $N = p_1 \cdot p_2 \cdot \ldots \cdot p_n + 1$.
Then $N$ is not divisible by any $p_i$, so either $N$ is prime or has a prime factor not in our list.
This contradicts our assumption.
\end{proof}

\begin{lemma}[Prime Divisibility]
If a prime $p$ divides a product $a \cdot b$, then $p \mid a$ or $p \mid b$.
\end{lemma}

\begin{theorem}[Fundamental Theorem of Arithmetic]
Every natural number $n > 1$ can be expressed uniquely as a product of prime powers.
\end{theorem}

\begin{example}
The number $12 = 2^2 \cdot 3$ has the unique prime factorization with primes $2$ and $3$.
\end{example}

\end{document}